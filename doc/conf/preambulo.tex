%% Preambulo LaTeX: Define classes e características do documento
%% Definição do docuemnto
\documentclass[
	12pt,           % Fonte: 12pt
	%openright,     % capítulos começam em página ímpar (use apenas se usar "twoside")
	oneside,        % Impressão: oneside = 1 face, twoside = 2 faces (frente-e-verso)
	a4paper,        % Tamanho do Papel: A4
	chapter=TITLE,  % Todos os capítulos devem ficam em caixa alta
	section=TITLE,  % Todas as seções devem ficar em caixa alta
	english,        % Adiciona Idioma para Hifenização: Inglês
	%spanish,       % Adiciona Idioma para Hifenização: Espanhol
	%french,        % Adiciona Idioma para Hifenização: Francês
	brazil          % Adiciona Idioma para Hifenização: Português do Brasil (o último idioma se torna o principal do documento)
]{abntex2}          % Utilizar ABNTeX2

%% Tipografia
%% Abra este arquivo e selecione uma das opções de fonte nele. A padrão é Times.
\input{conf/tipografia}

%% Pacotes usados pelo documento
%\usepackage{courier}                         % Permite a utilização da fonte Courier (para códigos-fonte)
\usepackage[T1]{fontenc}                      % Seleção de códigos de fonte.
\usepackage[utf8]{inputenc}                   % Codificação do documento (conversão automática dos acentos)
\usepackage{indentfirst}                      % Indenta o primeiro parágrafo de cada seção.
%\usepackage{nomencl}                         % Usado pela Lista de símbolos
\usepackage{color}                            % Controle das cores
\usepackage{graphicx}                         % Inclusão de gráficos
\usepackage{float}                            % Melhorias para posicionamento de gráficos e tabelas
\usepackage{microtype}                        % Melhorias na justificação
%\usepackage{lastpage}                        % Dá acesso ao número da última página do documento
%\usepackage{booktabs}                        % Comandos para tabelas
%\usepackage{multirow, array}                 % Múltiplas linhas e colunas em tabelas
%\usepackage[hyphenbreaks]{breakurl}          % Hifenação para URLs no texto
%\usepackage[table,xcdraw]{xcolor}            % Cores para algumas tabelas especiais
\usepackage[brazilian,hyperpageref]{backref}  % Inclui nas Referências as páginas onde há as citações
%\usepackage{simplecd}                        % Pacote para gerar capa do CD
%\usepackage[final]{pdfpages}                 % Pacote para incluir um PDF dentro de outro (ficha catalográfica)
%\usepackage[many]{tcolorbox}
%\usepackage{amsthm}

% ----------------------------------------------------------
% Normalização de fontes e tamanhos
%
% Retirado de: https://github.com/marbutto/tcc/blob/master/tcc.cls
% Autor: Matheus Liberatos, matheusliberatosbs@gmail.com
% ----------------------------------------------------------
\renewcommand{\ABNTEXchapterfont}{\sffamily}
\renewcommand{\ABNTEXchapterfont}{}
\renewcommand{\ABNTEXchapterfontsize}{\normalsize\bfseries}
\renewcommand{\ABNTEXpartfont}{\ABNTEXchapterfont}
\renewcommand{\ABNTEXpartfontsize}{\ABNTEXchapterfontsize}
\renewcommand{\ABNTEXsectionfont}{\ABNTEXchapterfont}
\renewcommand{\ABNTEXsectionfontsize}{\normalsize\bfseries}
\renewcommand{\ABNTEXsubsectionfont}{\ABNTEXsectionfont}
\renewcommand{\ABNTEXsubsectionfontsize}{\normalsize\bfseries}
\renewcommand{\ABNTEXsubsubsectionfont}{\ABNTEXsubsectionfont}
\renewcommand{\ABNTEXsubsubsectionfontsize}{\normalsize\bfseries}
\renewcommand{\ABNTEXsubsubsubsectionfont}{\ABNTEXsubsectionfont}
\renewcommand{\ABNTEXsubsubsubsectionfontsize}{\normalsize\bfseries}

% ----------------------------------------------------------
% Normalização de fontes e tamanhos no Sumário, Seções, etc.
% ----------------------------------------------------------
\renewcommand{\cftsectionfont}{\bfseries}
\renewcommand{\cftsectionpagefont}{\cftsectionfont}
\renewcommand{\cftsubsectionfont}{\normalsize}
\renewcommand{\cftsubsectionpagefont}{\cftsectionfont}
\renewcommand{\cftsubsubsectionfont}{\normalsize}
\renewcommand{\cftsubsubsectionpagefont}{\cftsectionfont}
\renewcommand{\cftparagraphfont}{\normalsize}
\renewcommand{\cftparagraphpagefont}{\cftsectionfont}
\renewcommand{\cftsectionfont}{}                     % Tira o negrito das seções no sumário.
\renewcommand{\cftsectionpagefont}{\cftsectionfont}  % Tira o negrito das seções no sumário.

% ----------------------------------------------------------
% Alterações da Capa (inclusão do Brasão)
% ----------------------------------------------------------
\renewcommand{\imprimircapa}{
	\begin{capa}
		\center

		% Brasão (mude a escala para diminuir/aumentar)
		\begin{figure}[H]
			\centering
			\includegraphics[scale=0.12]{conf/brasao-brasil.pdf}
		\end{figure}

		% Nome da Instituição
		{\ABNTEXchapterfont\imprimirinstituicao\vspace*{3.5cm}}

		% Autor do Trabalho
		{\ABNTEXchapterfont\MakeUppercase{\imprimirautor}\vspace*{3.5cm}}

		% Título do Trabalho
		{\ABNTEXchapterfont\bfseries\MakeUppercase{\imprimirtitulo}}
		\vspace*{\fill}

		% Local e Ano
		{\MakeUppercase{\imprimirlocal}}
		\par
		{\imprimirdata}

		\vspace*{1cm}
	\end{capa}
}

% ----------------------------------------------------------
% Folha de Rosto (contracapa)
% ----------------------------------------------------------
\renewcommand{\folhaderostocontent}{
	\begin{center}
		{\ABNTEXchapterfont\MakeUppercase{\imprimirautor}}

		\vspace*{3.5cm}

		{\ABNTEXchapterfont\MakeUppercase{\imprimirtitulo}}

		\vspace*{\fill}

		\hspace{.45\textwidth}
		\begin{minipage}{.5\textwidth}
			\SingleSpacing
			\imprimirpreambulo
		\end{minipage}

		\hspace{.45\textwidth}
		\begin{minipage}{.5\textwidth}
			\SingleSpacing
			\textbf{Orientador:} \imprimirorientador
		\end{minipage}
		\vspace*{\fill}

		\vspace*{\fill}
		{\MakeUppercase{\imprimirlocal}}
		\par
		{\imprimirdata}
		\vspace*{1cm}
	\end{center}
}

% ----------------------------------------------------------
% Comandos Novos para os Membros da Banca
% Utilize \membroum{Nome} e \membrodois{Nome} para configurar
% ----------------------------------------------------------
\newcommand{\membroumname}{Primeiro membro}
\newcommand{\membrodoisname}{Segundo membro}
\newcommand{\membrotresname}{Terceiro membro}
\newcommand{\dataapresentacaoname}{01 de Janeiro de 1970}
\newcommand{\nomedocursoname}{Nome do Curso}

% Comandos de dados - Membro da banca um
\providecommand{\imprimirmembroumRotulo}{}
\providecommand{\imprimirmembroum}{}
\newcommand{\membroum}[2][\membroumname]
	{\renewcommand{\imprimirmembroumRotulo}{#1}
		\renewcommand{\imprimirmembroum}{#2}}

% Comandos de dados - Membro da banca dois
\providecommand{\imprimirmembrodoisRotulo}{}
\providecommand{\imprimirmembrodois}{}
\newcommand{\membrodois}[2][\membrodoisname]
	{\renewcommand{\imprimirmembrodoisRotulo}{#1}
		\renewcommand{\imprimirmembrodois}{#2}}

% Comandos de dados - Membro da banca três
\providecommand{\imprimirmembrotresRotulo}{}
\providecommand{\imprimirmembrotres}{}
\newcommand{\membrotres}[2][\membrotresname]
	{\renewcommand{\imprimirmembrotresRotulo}{#1}
		\renewcommand{\imprimirmembrotres}{#2}}

% Comandos de dados - Data da apresentação
\providecommand{\imprimirdataapresentacaoRotulo}{}
\providecommand{\imprimirdataapresentacao}{}
\newcommand{\dataapresentacao}[2][\dataapresentacaoname]{\renewcommand{\dataapresentacao}{#2}}

% Comandos de dados - Nome do Curso
\providecommand{\imprimirnomedocursoRotulo}{}
\providecommand{\imprimirnomedocurso}{}
\newcommand{\nomedocurso}[2][\nomedocursoname]
	{\renewcommand{\imprimirnomedocursoRotulo}{#1}
		\renewcommand{\imprimirnomedocurso}{#2}}

% Comandos de dados - Instituição curto
\providecommand{\imprimirinstituicaocurto}{}
\newcommand{\instituicaocurto}[1]{\renewcommand{\imprimirinstituicaocurto}{#1}}

% ----------------------------------------------------------
% Folha de Aprovação
% ----------------------------------------------------------
\newcommand{\imprimirfolhadeaprovacao}{
\begin{folhadeaprovacao}
	\begin{center}
		{\ABNTEXchapterfont\MakeUppercase{\imprimirautor}}
		\vspace*{\fill}

		\vspace*{\fill}
		{\ABNTEXchapterfont\MakeUppercase{\imprimirtitulo}}
		\vspace*{\fill}

		\hspace{.45\textwidth}
		\begin{minipage}{.5\textwidth}
			\imprimirpreambulo
		\end{minipage}
		\vspace*{\fill}
	\end{center}

	\begin{center}
		Aprovado pela banca examinadora em \dataapresentacao.
	\end{center}

	\setlength{\ABNTEXsignwidth}{12cm}
	\setlength{\ABNTEXsignskip}{0.5cm}
	\assinatura{\textbf{\imprimirorientador}  \\ \imprimirinstituicaocurto}
	\assinatura{\textbf{\imprimirmembroum}    \\ \imprimirinstituicaocurto}
	\assinatura{\textbf{\imprimirmembrodois}  \\ \imprimirinstituicaocurto}
	%\assinatura{\textbf{\imprimirmembrotres} \\ \imprimirinstituicaocurto}

	\begin{center}
		\vspace*{\fill}
		{\MakeUppercase{\imprimirlocal}}
		\par
		{\imprimirdata}
		\vspace*{1cm}
	\end{center}

\end{folhadeaprovacao}
}

% ----------------------------------------------------------
% Folha de Aprovação - Duas Colunas (para 4 membros)
% ----------------------------------------------------------
\newcommand{\imprimirfolhadeaprovacaoduascolunas}{
\begin{folhadeaprovacao}
	\begin{center}
		{\ABNTEXchapterfont\MakeUppercase{\imprimirautor}}
		\vspace*{\fill}

		\vspace*{\fill}
		{\ABNTEXchapterfont\MakeUppercase{\imprimirtitulo}}
		\vspace*{\fill}

		\hspace{.45\textwidth}
		\begin{minipage}{.5\textwidth}
			\imprimirpreambulo
		\end{minipage}
		\vspace*{\fill}
	\end{center}

	\begin{center}
		Aprovado pela banca examinadora em \dataapresentacao.
	\end{center}

	\begin{center}
		\setlength{\ABNTEXsignwidth}{6cm}
		\setlength{\ABNTEXsignskip}{0.7cm}
		\begin{minipage}{.45\textwidth}
			\assinatura{\textbf{\imprimirorientador} \\ \imprimirinstituicaocurto}
		\end{minipage}
		\hfill
		\begin{minipage}{.45\textwidth}
			\assinatura{\textbf{\imprimirmembroum}   \\ \imprimirinstituicaocurto}
		\end{minipage}

		\begin{minipage}{.45\textwidth}
			\assinatura{\textbf{\imprimirmembrodois} \\ \imprimirinstituicaocurto}
		\end{minipage}
		\hfill
		\begin{minipage}{.45\textwidth}
			\assinatura{\textbf{\imprimirmembrotres} \\ \imprimirinstituicaocurto}
		\end{minipage}
	\end{center}

	\begin{center}
		\vspace*{\fill}
		{\MakeUppercase{\imprimirlocal}}
		\par
		{\imprimirdata}
		\vspace*{1cm}
	\end{center}

\end{folhadeaprovacao}
}

%% Metadados
%% Configurações dos metadados do PDF
\makeatletter
\hypersetup{
	pdftitle={\@title},
	pdfauthor={\@author},
	pdfsubject={\@title},
	pdfcreator={LaTeX, abnTeX2, {abnTeX\-ifpi}},
	%pdfkeywords={key}{key}{key}{key}{key}{key}, TODO
	colorlinks=true,     % Visual dos Links: false = caixas; true = colorido
	linkcolor=cor-link,  % Cor dos Links Internos (preto)
	citecolor=cor-link,  % Cor de Links para Bibliografia (preto)
	filecolor=cor-link,  % Cor para Links a Arquivos (preto)
	urlcolor=cor-link,   % Cor para Links a URLs (preto)
	bookmarksdepth=4
}
\makeatother


%% Metadados
%%%%%%%%%%%%%%%%%%%%%%%%%%%%%%%%%%%%%%%%%%%%%%%%%%%%%%%%%%%%%%%%%%%%%%%%%%%%%%%%
%% Metadados do trabalho
%% Esses dados serão automaticamente convertidos para caixa alta onde necessário
%%%%%%%%%%%%%%%%%%%%%%%%%%%%%%%%%%%%%%%%%%%%%%%%%%%%%%%%%%%%%%%%%%%%%%%%%%%%%%%%

%% Título
\titulo{Atarefado: Sistema Web para organização de lista de tarefas}

%% Autor
\autor{Breno Ramalho Lemes}

%% Local de publicação
\local{Lucas do Rio Verde}

%% Preâmbulo do trabalho
\preambulo{Trabalho de Conclusão de Curso apresentado à Universidade Aberta do Brasil no Curso de Tecnologia em Sistemas para Internet do Instituto Federal de Educação, Ciência e Tecnologia de Mato Grosso, como exigência para a obtenção do título de Tecnólogo.}

%% Orientador
%% "M\textsuperscript{e}." = Abreviação oficial para "Mestre"
\orientador{Valtemir Emerencio do Nascimento}

%% Tipo de Trabalho
%% - Monografia
%% - Tese (Mestrado)
%% - Tese (Doutorado)
%% - Relatório técnico
\tipotrabalho{Monografia}

%% Data do Trabalho
\data{2020}

%% Nome da Instituição (para a capa)
\instituicaocurto{Instituto Federal de Mato Grosso - IFMT}
\nomedocurso{Curso de Tecnologia em Sistemas para Internet}
\instituicao{\imprimirinstituicaocurto
\\
Universidade Aberta do Brasil
\\
\imprimirnomedocurso}

%% Primeiro membro da banca examinadora
\membroum{Avaliador Técnico}

%% Segundo membro da banca examinadora
\membrodois{Avaliador Metodológico}

%% Data da apresentação do trabalho
%% Se não souber a data da apresentação, utilize \underline{\hspace{3.5cm}}
%% Isso cria um sublinhado de 3.5cm, onde você pode escrever a data depois!
\dataapresentacao{\underline{\hspace{3.5cm}}}


%% Configuração do "Citado nas páginas"
\input{conf/citacoes}

%% Cores
\input{conf/cores}

%% Espaçamentos
\input{conf/espacamentos}
