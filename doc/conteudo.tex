
\chapter{Introdução}

Aplicações web utilizam-se do protocolo HTTP (Hypertext Transfer Protocol). Esse protocolo desde sua versão 1.1 é definido como ``um protocolo em nível de aplicativo para sistemas de informações hipermídia distribuídos, colaborativos.'' através do RFC 2616.

``Em computação, um protocolo sem estado (do inglês stateless) é um protocolo de comunicação que considera cada requisição como uma transação independente que não está relacionada a qualquer requisição anterior, de forma que a comunicação consista de pares de requisição e resposta independentes.'' \url{https://pt.wikipedia.org/wiki/Protocolo_sem_estado}

Para criar aplicações web queremos justamente o inverso: manter um estado. É necessário saber as ações que o usuário faz entre uma requisição e outra. E, ainda mais importante, identificar o usuário, mantendo essa identidade entre as requisições. Visando garantir a confidencialidade das informações tratadas pela aplicação. O recurso que utilizamos para manter um estado entre uma requisição e outra de um protocolo sem estado é o gerenciamento de sessão.

% ``Em computação, aplicação web designa, de forma geral, sistemas de informática projetados para utilização através de um navegador, através da internet ou aplicativos desenvolvidos utilizando tecnologias web HTML, JavaScript e CSS.'' \url{https://pt.wikipedia.org/wiki/Aplicação_web}

% Pela sua natureza de utilização multi-usuário, especialmente através da internet com usuários podendo acessar de qualquer parte do mundo, é necessário identificar o usuário para garantir a confidencialidade das informações tratadas pela aplicação.
% As aplicações precisam

% ``O gerenciamento de sessão é o processo de manter informações de estado entre várias solicitações''. Xue Bai


% Protocolo HTTP stateless.

\section{Objetivos}

O objetivo deste trabalho é estudar as técnicas de gerenciamento existentes e as vantagens e desvantagens de cada uma delas.

Por meio desse estudo, escolher a técnica mais adequada e implementar uma biblioteca de gerenciamento de sessão com a linguagem de programação Lua para uso no servidor CivetWeb.

% IMPLEMENTAÇÃO baseada nos servlet Java

\section{Justificativa}

\begin{quote}
``Tudo deve ser feito tão simples quanto possível; mas não mais simples.''
Albert Einstein
\end{quote}

O CivetWeb é um servidor web pequeno, leve e fácil de usar. Um único executável com menos que 1,5MB oferece suporte a banco de dados SQLite e linguagem Lua. Também é multi-plataforma (Windows, Linux, e outros). É um servidor ideal para sistemas que possuem restrições, seja de capacidade de processamento, armazenamento e/ou memória. Também não tem dependência de outros componentes, como Framework .NET ou JRE (Java Runtime Environment).

No entanto o CivetWeb não oferece pronto para uso funções para gerenciamento de sessão. Considerando que esse é um recurso essencial para garantir a segurança das aplicações, é necessário buscar uma biblioteca de terceiros para esse fim. Esse trabalho visa justamente atender essa necessidade, tornando o uso do CivetWeb com linguagem Lua uma opção um pouco mais completa e viável para desenvolver aplicações web.

% Muitos ambientes corporativos possuem restrições quanto a que softwares podem ou não serem instalados. Nem sempre possuem componentes necessários para que determinadas ferramentas de desenvolvimento funcionem, como Framework .NET ou JRE (Java Runtime Environment). O que pode acabar inviabilizando a implantação de um projeto.

% Alguns equipamentos possuem restrições quanto a capacidade de processamento, armazenamento e/ou memória. Tornando as ferramentas mais populares de desenvolvimento inviáveis para determinados projetos, como por exemplo equipamentos embarcados.

% Cenários com usuários e/ou clientes com pouco conhecimento técnico podem inviabilizar tarefas complexas como instalação, configuração e manutenção, de servidores web, bancos de dados, etc.

% Por isso a escolha por um sistema simples, pequeno, multi-plataforma (Windows, Linux, e outros). O servidor CivetWeb, com banco de dados SQLite, e linguagem Lua. Tudo em menos que 1,5MB para os três componentes. Poderia ser instalado por exemplo até mesmo num roteador, para compor o sistema de configuração via Web.

% \chapter{Especificação de requisitos}

Teste de citação \cite{cnet-google-crowdsourcing}.

% \chapter{Modelagem de dados}

% \chapter{Considerações sobre segurança}

% \section{Firewall}

% \section{Autenticação HTTP Basic}

% \section{Autenticação HTTP Digest}

% \section{Autenticação baseada em Token}

% \chapter{Servidor CivetWeb}

% \chapter{Linguagem de programação Lua}

% \chapter{Banco de dados SQLite}

% \chapter{Interface gráfica com HTML, CSS e JavaScript}

\chapter{Conclusão}
