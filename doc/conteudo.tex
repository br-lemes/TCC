
\chapter{Introdução}

Uma lista de tarefas é uma excelente ferramenta para ajudar a alcançar objetivos.

Existe uma grande quantidade de aplicativos e sistemas em nuvem para realizar lista de tarefas. O que diferencia este dos demais, é que você controla os seus dados. Não estão na nuvem. Por ser um sistema de código aberto, você instala no servidor de sua preferência, seja na Internet, numa Intranet, no seu próprio computador, num dispositivo móvel (Android por exemplo), numa placa de desenvolvimento (Raspberry Pi por exemplo), etc. O banco de dados é um arquivo SQLite, sobre o qual você escolhe um método de backup e/ou sincronização. Você toma as decisões de onde colocar o seu sistema e onde armazenar os seus dados.

\section{Objetivos}

Criar um sistema de lista de tarefas.

\begin{enumerate}
	\item Especificação de requisitos
	\item Modelagem de dados
	\item Considerações sobre segurança
	\item Descrever as tecnologias utilizadas:
	\begin{enumerate}
		\item Servidor CivetWeb
		\item Linguagem de programação Lua
		\item Banco de dados SQLite
		\item Interface gráfica com HTML, CSS e JavaScript
	\end{enumerate}
\end{enumerate}

\section{Justificativa}

\begin{quote}
``Tudo deve ser feito tão simples quanto possível; mas não mais simples.''
Albert Einstein
\end{quote}

Muitos ambientes corporativos possuem restrições quanto a que softwares podem ou não serem instalados. Muitas vezes não possuem componentes necessários para que determinadas ferramentas de desenvolvimento funcionem, como Framework .NET ou JRE (Java Runtime Environment). Por vezes inviabilizando a implantação de um projeto.

Muitos equipamentos possuem restrições quanto a capacidade de processamento, armazenamento e/ou memória. Tornando as ferramentas mais populares de desenvolvimento inviáveis para determinados projetos, como por exemplo equipamentos embarcados.

Muitas vezes você precisa lidar com usuários e/ou clientes com pouco conhecimento técnico, inviabilizando tarefas complexas como instalação, configuração e manutenção, de servidores web, bancos de dados, etc.

Por isso a escolha por um sistema simples, pequeno, multi-plataforma (Windows, Linux, e possivelmente outros). O servidor CivetWeb, com banco de dados SQLite, e linguagem Lua. Tudo em menos que 1,5MB para os três componentes. Poderia ser instalado por exemplo até mesmo num roteador, para compor o sistema de configuração via Web.

\chapter{Especificação de requisitos}

Teste de citação \cite{cnet-google-crowdsourcing}.

\chapter{Modelagem de dados}

\chapter{Considerações sobre segurança}

\section{Firewall}

\section{Autenticação HTTP Basic}

\section{Autenticação HTTP Digest}

\section{Autenticação baseada em Token}

\chapter{Servidor CivetWeb}

\chapter{Linguagem de programação Lua}

\chapter{Banco de dados SQLite}

\chapter{Interface gráfica com HTML, CSS e JavaScript}

\chapter{Conclusão}
