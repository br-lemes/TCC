
\chapter{Introdução}

Uma lista de tarefas é uma excelente ferramenta para ajudar a alcançar objetivos.

Existe uma grande quantidade de aplicativos e sistemas em nuvem para realizar lista de tarefas. O que diferencia este dos demais, é que você controla os seus dados. Não estão na nuvem. Por ser um sistema de código aberto, o usuário instala no servidor de preferência dele, seja na Internet, numa Intranet, no próprio computador, num dispositivo móvel (Android por exemplo), numa placa de desenvolvimento (Raspberry Pi por exemplo), etc. O banco de dados é um arquivo SQLite, sobre o qual o usuário escolhe um método de backup e/ou sincronização. É o usuário quem toma as decisões de onde colocar o sistema dele e onde armazenar os dados dele.

\section{Objetivos}

Criar um sistema de lista de tarefas.

\begin{enumerate}
	\item Especificação de requisitos
	\item Modelagem de dados
	\item Considerações sobre segurança
	\item Descrever as tecnologias utilizadas:
	\begin{enumerate}
		\item Servidor CivetWeb
		\item Linguagem de programação Lua
		\item Banco de dados SQLite
		\item Interface gráfica com HTML, CSS e JavaScript
	\end{enumerate}
\end{enumerate}

\section{Justificativa}

\begin{quote}
``Tudo deve ser feito tão simples quanto possível; mas não mais simples.''
Albert Einstein
\end{quote}

Muitos ambientes corporativos possuem restrições quanto a que softwares podem ou não serem instalados. Nem sempre possuem componentes necessários para que determinadas ferramentas de desenvolvimento funcionem, como Framework .NET ou JRE (Java Runtime Environment). O que pode acabar inviabilizando a implantação de um projeto.

Alguns equipamentos possuem restrições quanto a capacidade de processamento, armazenamento e/ou memória. Tornando as ferramentas mais populares de desenvolvimento inviáveis para determinados projetos, como por exemplo equipamentos embarcados.

Cenários com usuários e/ou clientes com pouco conhecimento técnico podem inviabilizar tarefas complexas como instalação, configuração e manutenção, de servidores web, bancos de dados, etc.

Por isso a escolha por um sistema simples, pequeno, multi-plataforma (Windows, Linux, e outros). O servidor CivetWeb, com banco de dados SQLite, e linguagem Lua. Tudo em menos que 1,5MB para os três componentes. Poderia ser instalado por exemplo até mesmo num roteador, para compor o sistema de configuração via Web.

\chapter{Especificação de requisitos}

Teste de citação \cite{cnet-google-crowdsourcing}.

\chapter{Modelagem de dados}

\chapter{Considerações sobre segurança}

\section{Firewall}

\section{Autenticação HTTP Basic}

\section{Autenticação HTTP Digest}

\section{Autenticação baseada em Token}

\chapter{Servidor CivetWeb}

\chapter{Linguagem de programação Lua}

\chapter{Banco de dados SQLite}

\chapter{Interface gráfica com HTML, CSS e JavaScript}

\chapter{Conclusão}
